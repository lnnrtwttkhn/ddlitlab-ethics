\documentclass[11pt,twoside,a4paper]{article}
\usepackage{style}
\input{config}
\titleformat{\section}{\fontsize{12pt}{14pt}\selectfont\bfseries}{\thesection}{1em}{}
\titleformat*{\subsection}{\normalsize\bfseries\color{black}}
\titleformat*{\subsubsection}{\normalsize\bfseries\color{black}}
\titleformat*{\paragraph}{\normalsize\bfseries\color{black}}
\titleformat*{\subparagraph}{\normalsize\bfseries\color{black}}

% enable section numbering:
\setcounter{secnumdepth}{3} 
\title{\vspace{-1.5cm}\Large General Information for Participants
\vspace{-1.7cm}}
\lfoot{Template for the General Participant Information as of April 20, 2021 (according to the template provided by the German Psychological Society (DGPs) on June 20, 2017). Translated from German to English.}
\lhead{Ethics Proposal~--~\ApplicantName~--~\DateApplication~--~Version \HeadSha\newline{}}
\lhead{Ethics Proposal~--~Dr.~Lennart Wittkuhn~--~\today\newline{}General participant information about the study}


\begin{document}
\thispagestyle{fancy}

\begin{minipage}{0.5\linewidth}
    \includegraphics[width=\linewidth]{uhh-logo.pdf}
\end{minipage}
\begin{minipage}{0.5\textwidth}\raggedleft{}
    \ApplicantUniversity{}, \ApplicantInstitute{}\\
    \ApplicantResearchGroup{}\\
    Contact person for any inquiries:\\
    Project Leader: \ProjectLeaderName{}\\
    Telephone: \InvestigatorPhoneNumber{}\\
    Email: \InvestigatorEmailAddress{}
\end{minipage}
\vspace*{1em}

\begin{flushleft}
    \Large\textbf{General information for participants}
\end{flushleft}

\section{Title of the study: \ProjectTitle{}}

Welcome to our study ``\ProjectTitle{}''!
We thank you for your interest in this study.

We are investigating with this study, \hl{>etc.<}.

You will find more information about this study in the following sections.
We kindly ask you to read the text carefully and calmly.
If you have any questions or if there is something that you do not understand, please do not hesitate to ask us.
We are happy to provide further clarification or additional information.

\section{Study Procedure}

The following experiment consists of \hl{>information about durations and breaks<}.
The total duration of the experiment is \hl{>xx<} hours.
Your task is to \hl{>etc.<}.
\hl{>Here is what the participants need to do<.}
\hl{>Additional information, such as wearing earplugs, headphones, giving instructions, intercom, rating scales, response boxes, etc.<}.

Should there be any issues or concerns during the study, please contact the study investigator.

\section{Voluntariness and anonymity}

Participation in the study is voluntary.
You can withdraw from the study at any time and without giving any reasons, without any disadvantages to you.
Even if you terminate the study prematurely, you are entitled to \hl{appropriate compensation / corresponding research participant hours} for the time already spent (see section \ref{sec:renumeration}).

The data and personal information collected within this study, as described above, will be treated confidentially.
Project team members who have direct access to your personal data are bound by confidentiality.
Furthermore, the publication of the study results will be in an anonymized form, meaning that your data will not be linked to your identity.

\section{Data protection}
\label{sec:data_protection}

The collection of your personal data, as described above, will be entirely anonymized, meaning that your name will not be requested at any point.
Your answers and results will be stored under a personal code word that you have created based on a set of rules, known only to you.
This means that no one else can link your data to your name.
The anonymized data will be stored for at least 10 years.
However, you have the option to request the deletion of the data collected from you at any time.
To do so, you do not need to disclose your name.
Only your code word is required.
You will receive instructions on ``How to create your personal code word?''.
This document will remain with you, so please keep it safe so that you can request the deletion of your data if necessary in the future.

\section{Extent of Data Collection and Processing}

The data can only be used for the purpose of the research project.
Your data

\hl{[The purpose should be specified in a way that provides an overview of the scope of the collected data.}
\hl{If the exact use of the data is not yet known, it should include research areas.}
\hl{In general, the purpose should be broad enough to avoid the need for subsequent re-consent for data collection.}
\hl{However, a blanket consent is not possible.}
\hl{If there are multiple research areas and/or research questions, all of them must be listed, and an active option for selection (consent, rejection, checking options (opt-in procedure), Art. 4 No. 11 GDPR) must be provided.}
\hl{If the responsible party wishes to process the data for a purpose other than the original one, they must inform the data subjects before further processing and obtain their consent.]}

The information must also clearly state:

Data to be collected: What personal data will be collected?

This includes both the data collected for the actual purpose of the studies (e.g., questionnaires, reaction times) and any additional personal or demographic data such as name, age, gender, etc.

Analysis results from the data: What analyses will be derived from the data or have the potential to be derived?
Do the analyses potentially reveal sensitive data?

Are there other recipients or categories of recipients: To which recipients will the personal data be transmitted?
What are the purposes of such transmission?

\section{Retention Period for Anonymized Data}

The fully anonymized data will be made publicly accessible through internet databases (including, \href{https://github.com/}{github.com}, \href{https://osf.io/}{osf.io}, \href{https://www.seafile.com/en/home/}{seafile.com}, \href{https://gin.g-node.org/}{gin.g-node.org}, or \href{https://cloud.uni-hamburg.de/}{Nextcloud}).
This will be done in an anonymized form, meaning that the data will not be linked to any specific individual.
This approach follows the recommendations of the German Research Foundation (DFG) and the German Psychological Society (DGPs) for ensuring research quality.
This procedure ensures good scientific practice.
Other researchers can, for instance, replicate the analysis or test alternative evaluations using this data.

\section*{Your Rights}

According to Art.~13(2)(b) of the General Data Protection Regulation (GDPR), you have the right to:

\begin{enumerate}
  \item \textbf{Access (Art. 15 GDPR and §34 BDSG)}:
  You have the right to request information about the data processed regarding your person, including potential recipients of this data.
  You are entitled to receive a response within one month from the date of your request.
  \item \textbf{Rectification, Erasure, and Restriction (Art. 16--18 GDPR and §35 BDSG):}
  You can request the correction, erasure, or restriction of the processing of your personal data by the \ApplicantUniversity, as long as your data can still be linked to your identity (see above).
  \item \textbf{Data Portability (Art. 20 GDPR):}
  You have the right to receive the personal data concerning you, which you have provided to a controller, in a structured, commonly used, and machine-readable format, where such data can be associated with your identity.
  \item \textbf{Objection (Art. 21 GDPR and §36 BDSG):}
  You have the right to withdraw your consent at any time with future effect.
  This can be done orally or by email.
  If necessary, you may be asked to verify your identity.
  Upon receipt of your statement, your data may no longer be processed.
  They must be deleted without delay. However, the previous processing remains unaffected.
\end{enumerate}

If you wish to exercise any of these rights, please contact the project leader:

Project Leader: \ProjectLeaderName\\
\ProjectLeaderUniversity, \ProjectLeaderInstitute, \ProjectLeaderResearchGroup\\
Address: \ProjectLeaderAddress\\
Email: \ProjectLeaderEmailAddress\\
Telephone: \ProjectLeaderPhoneNumber\\
Fax: \ProjectLeaderFaxNumber{}

You also have the right to file a complaint with the supervisory authority:
The competent state authority for data protection in Hamburg can be found on the service portal: \url{https://datenschutz-hamburg.de/}

Controller responsible for the processing of your data is:

The President of the University of Hamburg\\
Mittelweg 177\\
20148 Hamburg\\
\href{mailto:praesident@uni-hamburg.de}{praesident@uni-hamburg.de}

The responsible data protection officer is:

Data Protection Officer of the University of Hamburg\\
Mittelweg 177\\
20148 Hamburg\\
\href{mailto:datenschutz@uni-hamburg.de}{datenschutz@uni-hamburg.de}

\section{Renumeration}
\label{sec:renumeration}

\paragraph{Cash Payment Option:}

For participating in the study, you will receive compensation of \hl{>xx \euro{} per hour<}.
The payment will be given to you in cash.
Upon receiving the cash payment, \hl{>we will record your name / you will sign a receipt with your name (and address)<}.
This information will be stored separately from other data collected from you and will serve as proof for potential expenditure audits.
It will be deleted no later than \hl{>the date specified in the application<}.

\paragraph{Bank Transfer Option:}

For participating in the study, you will receive compensation of \hl{xx \euro{}  per hour<}.
The payment will be transferred to your bank account.
To do this, you need to provide your bank details.
All related information will be kept entirely separate from the study data and will be deleted immediately after the transfer.

\paragraph{Research Participant Hours Option:}

You can also choose to receive a credit of research participant hours equivalent to the time spent.
For participating in this study, you will be credited with \hl{>n<} research participant hours.

\end{document}
