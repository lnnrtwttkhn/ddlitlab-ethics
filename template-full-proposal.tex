\documentclass[11pt,twoside,a4paper]{article}
\usepackage{style}
\input{config}
\title{\vspace{-1cm}Request for Opinion of the Ethics Committee of the Faculty of Psychology and Human Movement Science of the University of Hamburg\vspace{-1.5cm}}
\lhead{Ethics Proposal~--~\ApplicantName{}~--~\DateApplication{}~--~\Version{}}
\titleformat{\section}{\fontsize{12pt}{14pt}\selectfont\bfseries\color{blue}}{\thesection}{1em}{}
\titleformat*{\subsection}{\normalsize\bfseries\color{black}}
\titleformat*{\subsubsection}{\normalsize\bfseries\color{black}}
\titleformat*{\paragraph}{\normalsize\bfseries\color{black}}
\titleformat*{\subparagraph}{\normalsize\bfseries\color{black}}
\begin{document}

\maketitle
\thispagestyle{fancy}

\section{Title of project}

``\ProjectTitle{}''

\section{Name and address of applicant (office address)}

\textbf{Name:} \ApplicantName{}\\
\textbf{Institution:} \ApplicantUniversity{}, \ApplicantInstitute{}, \ApplicantResearchGroup{}\\
\textbf{Address:} \ApplicantAddress{}\\
\textbf{E-Mail Address:} \ApplicantEmailAddress{}\\
\textbf{Phone Number:} \ApplicantPhoneNumber{}\\
\textbf{Fax Number:} \ApplicantFaxNumber{}

\section{Context of research}

% This is an application for funding by >funding institution<.
% An ethical review and statement by the ethics committee is required / not required.
% It is a monocentric/multicentric study.
% Other institutions involved in the study are:

\subsection{Research sponsor and funding of the project}

This study will be conducted within the \ApplicantResearchGroup{} (Project Leader: \ApplicantName{}; Group Leader: Prof.~Dr.~Nicolas Schuck), located at the \ApplicantInstitute{} at the \ApplicantUniversity{}, Germany.
This is a mono-centric study that is being conducted as part of the project ``\ProjectTitle{}''.
This project is funded by the \href{https://www.isa.uni-hamburg.de/en/ddlitlab.html}{Digital and Data Literacy in Teaching Lab (DDLitLab)}, an initiative by the Center for Interdisciplinary Study Programs (``Zentrum für interdisziplinäre Studienangebote''; \href{https://www.isa.uni-hamburg.de/}{ISA-Zentrum}) at the University of Hamburg.
The Digital and Data Literacy in Teaching Lab program is in turn funded by the \href{https://stiftung-hochschullehre.de/en/}{Stiftung Innovation in der Hochschullehre}.
Details of the funding initiative can be found \href{https://stiftung-hochschullehre.de/projekt/ddlitlab/}{here}.
There are no other institutions involved in the study.

\subsection{Is an ethics committee opinion required?}

We approach the Local Ethics Committee (LEC) of the Faculty of Psychology and Movement Science at the University of Hamburg for a statement about this study by our own accord.
It is of our top-most priority to ensure that the study design aligns with ethical principles.

\section{Subject and procedure of the project}

\subsection{Research subject}
% Specify research objective, see Instructions for Submitting Applications (Paragraph 4).


In the evaluation of their course on ``digital open science'', \citeA{Toelch2018} concluded that a limited number of sessions per topic, resulted in merely surface-level or excessively swift coverage of the material.
\citeA{Toelch2018} also assessed implementation success after the course by asking students half a year after completion of the course whether they had changed their behavior in the direction of open science.
While they found that more than half of the responding students planned to or actually engaged in open science practices, the topic of version control was a notable exception.
In this case, the evaluation revealed that students reported hesitancy in using it, with 35\% of them even indicating no intention of using it in the future.
\citeA{Toelch2018} advocate that teaching version control (1) requires more time, (2) should focus on students that will likely engage in larger software projects, and (3) should be part of any advanced open science data analysis course.
In particular, enough time should be allocated for learning more advanced version control techniques such as branching, merging, and pull requests.

\subsection{Methods}
% State main methods of investigation, e.g., measurement of reaction times, acquisition of EEG, completion of questionnaires.

\subsection{Sample size}
% Provide a detailed rationale for sample size including a power analysis if applicable, otherwise provide other detailed justification

\subsection{Experimental tasks}
% Describe details of the experimental tasks here; what are the subjects instructed to do?

\subsection{Procedure}

% Describe details of the study procedure here

\subsection{Physical stress}

This study does not involve any physical stress.

\subsection{Mental stress}

We expect mental stress to be minimal.
All surveys consist of non-offensive text-based visual input.
Participants can take breaks whenever necessary and control the duration of the breaks.
Furthermore, participants are fully informed about the study procedure and their right to discontinue participation at any time prior to the assessments.

\subsection{Disclosure of personal information}

We will not collect any personal data (name, age, gender, regular medication, or any other personal data) from participants.

\subsection{Deception and debriefing}

All study participants are fully informed about the background and objectives of the research project.
They will not be deliberately provided with incomplete or false information regarding the study goals and procedures.

\section{Collection, protected storage, and deletion of data}

Participants are informed in detail about data privacy in the information sheet and consent form and are provided the opportunity to ask questions.

\subsection{Personal data}

All data are collected in an anonymized form using a coding system (for details, see section \ref{sec:data_protection}).
We will not collect any personal data (name, age, gender, regular medication, or any other personal data) from participants.
Participants retain the right to withdraw from the study and request the deletion of their data (refer to the consent form in \hl{Appendix B}).

\subsection{Data protection}
\label{sec:data_protection}

To ensure the anonymity of the participants, all data will be processed only in connection with a personal code word.
In the study, participants will generate a code word immediately after agreeing to the informed consent.
For this purpose, participants are asked to create an eight-character code word by combining, the last two letters of their mother's maiden name, the number of letters of their mother's (first) given name, the last two letters of their father's (first) given name, and the day of their own birthday (for example, \texttt{ER04LF09}).
Detailed instructions for creating the code word are provided in \hl{Appendix B}.
This code word will be used in all measurements instead of any other identification features.
In the event of revoking the informed consent during or after the project duration, the data already collected will be deleted upon the request of the participants by providing their code word to the research team.

In a second step, after completing data collection, a second code list will be created, connecting the personal code words of the participants with a randomized number combination.
The personal code words will be replaced in the dataset, allowing fully anonymized data to be made available to the research community via internet databases (for details, see below).
This second code list will also only be accessible to the project team.
This ensures that participants have the option to request the deletion of their data even after the data collection is completed, using their personal code word.

\subsection{Confidentiality / obligation to data secrecy / non-disclosure}

The data collected as part of the study, after obtaining the consent of the study participants, is subject to confidentiality and data protection regulations.
The researchers involved in the study are legally bound to confidentiality and will be bound to data secrecy.
No third parties (e.g., physicians or teachers) have to be released from their duty of confidentiality or their obligation to data secrecy by the participants in the context of the study.
\hl{Are participants in group settings explicitly asked to maintain confidentiality with regard to personal information disclosed by other participants?}

\paragraph{Deletion of the data} 

% >Information on data deletion with and without request. Period of retention of data that cannot be fully anonymized.<

At all times, the data will only be identifiable through the personal code word, known only to the participants themselves.
No personal data or contact details will be stored.
To enable future reanalysis and adhere to good scientific practices, the anonymized data will not be deleted without request even after the publication of the results.
Participants have the option to request the deletion of their data even after the data collection is completed, using their personal code word (for details, see section \ref{sec:data_protection}).

\section{Recruitment and Compensation of Participants.}

\subsection{Recruitment}

Participants will be recruited worldw

\subsection{Sample from a database?}

Participants will not be recruited from databases.

\subsection{Sample characteristics}

There are no predefined sample characteristics.
Anyone can participate in the study at any time.

\subsection{Inclusion and exclusion criteria}

There are no inclusion or exclusion criteria.

\subsection{Internet-based data collection}

% How is compliance with inclusion and exclusion criteria ensured?
% Are contact persons available for the subjects in a timely manner?

Data will be exclusively collected via the internet.
As there are no inclusion or exclusion criteria, anyone can participate in the study at any time.
In case of any questions, the project leader is available for the participants via email or phone.

\subsection{Participation compensation}

% Compensation e.g., in money or participant-hours?
% Amount; method of payment

Participants receive no compensation.

\section{Voluntariness and withdrawal of consent}

\subsection{Voluntariness}

% Indicate measures to ensure voluntariness, e.g., participant information, time to decide whether to participate, avoidance of special benefits for participation

Participation is entirely voluntary.
We will communicate to participants that study participation will be entirely voluntary, and that they may (a) refuse to participate in the study at any point in time, and (b) have their data deleted (see \hl{above}).
The participants are informed about this and are given sufficient time to familiarize themselves with the details of the study.
They must confirm that they have understood all information about the investigation and have no further questions.
For all steps of the study, participants have to declare their consent in written form.
This is done through several ``Yes/No'' questions before the start of each assessment.
The consent form stresses the voluntary nature of participation.

\subsection{Withdrawal}

% Safeguarding the possibility of withdrawal at any time without disadvantages and the right to delete one's own data until the data is anonymized.

Participants can withdraw from the study and revoke their consent at any time without providing reasons.
No negative consequences will arise because of a revocation of consent.
In the event of withdrawal of consent during the project period, the data collected up to that point will be deleted.
In this case, the participant will dissolve the pseudonymization by disclosing the personal code word, as the dataset cannot be identified otherwise.
A withdrawal after the project is completed and the data has already been anonymized is not possible.
After this period, the code list will be deleted and no personal data will be attached to the data.

\section{Handling of Abnormal Signs (e.g., EEG or MRT)}

\subsection{Clarification}

% How is the patient informed about abnormal findings, e.g.~in EEG, MRI or test diagnostic examinations?

No diagnostics or related procedures that allow for the identification of abnormalities will be used.

\subsection{Restriction of participation}

% Is it stated in the participant information that the subject can only participate in the examination if he/she agrees to be notified of any abnormal findings?
% Is this consent obtained in the Informed Consent Form?
% See templates for General Participant Information and Informed Consent.

No diagnostics or related procedures that allow for the identification of abnormalities will be used.

\section{Informed Consent}

\subsection{Informedness}

% Is the principle of complete information respected?
% If not, what is the justification for incomplete information (deception) of the subjects?
% How will the subjects be debriefed after the study (attach wording)?
% Exactly what information will be given to subjects?
% General and possibly specific participant information (e.g., for EEG, MRI, TMS studies) should be attached to the proposal; templates for these are available for download.

The participants are fully informed about the background and objectives of the research project (see \hl{Appendix B}).
All procedures will be explained in a way appropriate for lay people.
Furthermore, the voluntary nature of participation will be emphasized and any remaining open questions can be clarified with the project leader via email or phone.

\subsection{Consent}

% After informing the subjects, their consent is obtained.
% Does the consent form contain all the necessary components (voluntary, informed, fully understood, possibility of withdrawal without disadvantages, signatures if consent form is printed)?
% In addition, there may be other components, e.g., consent to specific research methods.
% The consent form must be attached to the proposal; a template for this is available for download.<

Before each assessment commences, the participants are asked whether they have read the participant information and agree to it.
This is done through a ``Yes / No'' question and is mandatory for participation in the study.
The consent form stresses the voluntary nature of participation.

\subsection{Audio and video recordings}

% A separate declaration of consent must be obtained if audio and video recordings are to be made; a template for this is available for download.
% If no: There are valid reasons for not involving the local ethics committee. I/we will explain the reasons to the Commission Chair in my/our cover letter.

Neither audio nor video data will be recorded.

\section{Disclosure of possible conflict of interest of applicant or other persons involved in the study}

There are no conflicts of interest.

\section{Previous ethical review concerning this project}

The research project described in this proposal has already been reviewed by an ethics committee. 

\YesNo{}

If yes: The relevant ethics vote is attached to the application.

\section{Date and Signature}

\vspace{8ex}
\noindent\begin{tabular}{ll}
Hamburg, \today & Lennart Wittkuhn \\
\makebox[7cm]{\hrulefill} & \makebox[7cm]{\hrulefill}\\
Place and Date & Signature of Applicant\\[8ex]
\end{tabular}

\bibliography{references.bib}
\clearpage

\end{document}
